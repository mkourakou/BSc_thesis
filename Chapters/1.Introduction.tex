\chapter{Εισαγωγή} \label{introduction}

Το πρώτο κεφάλαιο είναι εισαγωγικό και ο σκοπός του είναι να εξηγήσει το κίνητρο της μελέτης που παρουσιάζεται στην παρούσα εργασία και να χαρτογραφήσει τα βήματα που ακολουθούνται.



\section{Κίνητρο}

Ενδιαφερόμαστε για την μεταβλητότητα των \textlatin{AGN} στις ακτίνες Χ και συγκεκριμένα την μη-περιοδική (στοχαστική) μεταβλητότητα. Μελέτες κοντινών ενεργών γαλαξιακών πυρήνων υποδεικνύουν συσχέτιση της στοχαστικής αυτής μεταβλητότητας με αστροφυσικά χαρακτηριστικά των \textlatin{AGN}. \\
Θα εξετάσουμε τρόπους να μετρήσουμε την μεταβλητότητα αυτή και την συγκρίνουμε με την φωτεινότητα στο ενεργειακό παράθυρο που μελετάμε, καθώς η φωτεινότητα είναι ένα σαφές παρατηρησιακό μέγεθος που εξάγεται χωρίς αστροφυσική μοντελοποίηση. \\
Θα μελετήσουμε την μέση μεταβλητότητα για συλλογές ενεργών γαλαξιακών πυρήνων ενός πληθυσμού και την σχέση της με τα στατιστικά χαρακτηριστικά των συλλογών. Αυτό είναι ιδιαίτερα χρήσιμο για μακρόχρονες έρευνες όπως αυτές του \textlatin{XMM-Newton}, στις οποίες έχουμε αραιές και άτακτες παρατηρήσεις μεγάλου πληθυσμού πηγών για τις οποίες λόγω της στοχαστικότητας που τις διέπει μπορούμε να υπολογίσουμε τον μέσο όρο και να τις μελετήσουμε στατιστικά.\\
Θα δούμε ότι οι θεμελιώδεις φυσικές ποσότητες που χαρακτηρίζουν ένα σύστημα προσαύξησης είναι η μάζα μελανής οπής και ο ρυθμός προσαύξησης (η μάζα μελανής οπής υπαγορεύει σχεδόν όλες τις χαρακτηριστικές χρονικές κλίμακες του συστήματος, ενώ τόσο η μάζα μελανής οπής όσο και ο ρυθμός προσαύξησης εμφανίζονται στις πρωτης τάξεως διαταραχές λαμπρότητας τόσο στο μοντέλο των λεπτών δίσκων προσαύξησης όσο και των αδρών δίσκων \textlatin{ADAF}). Χρησιμοποιούμε ήδη υπάρχοντα παρατηρησιακά μοντέλα που εξετάζουν την επίδραση των θεμελιωδών ποσοτήτων αυτών στο φάσμα και συνεπώς στην μεταβλητότητα των ενεργών γαλαξιακών πυρήνων και τα συγκρίνουμε με τα αποτελέσματά μας για το πεδίο \textlatin{XMM-XXL-North}. \\
Έχοντας μια καταληκτική σχέση της μεταβλητότητας με την φωτεινότητα και εφαρμόζοντας αστροφυσική μοντελοποίηση για τα θεμελιώδη χαρακτηριστικά των \textlatin{AGN} μπορούμε να εξάγουμε αποτελέσματα που θέτουν παρατηρησιακά όρια (\textlatin{constraints}) για τα συστήματα προσάυξησης και κατ> επέκταση μας βοηθούν στην κατανόηση της ίδιας της διαδικασίας προσαύξησης.

\section{Περίγραμμα εργασίας}

Τα ακόλουθα κεφάλαια της παρούσας εργασίας έχουν δομηθεί ως εξής:

Κεφάλαιο 2: Γινεται μια σύνοψη των παρατηρησιακών χαρακτηριστικών των ενεργών γαλαξιακών πυρήνων σε διαφορετικά μήκη κύματος, πώς αυτά συνθέτουν κατηγοριακά τα αντικείμενα που ονομάζουμε ενεργούς γαλαξιακούς πυρήνες και πώς αυτά μας οδηγούν στο καθιερωμένο μοντέλο της υπερμεγέθους μελανής οπής με δίσκο προσάυξησης. Τα θεμελιώδη φυσικά χαρακτηριστικά του μοντέλου αυτού σχολιάζονται σύμφωνα με την σύγχρονη γνώση των συστημάτων προσάυξησης.

Κεφάλαιο 3: Περιγράφονται οι διαδικασίες ακτινοβολίας στην αστροφυσική όπως θεμελιώνονται από αρχές ηλεκτρομαγνητισμού, σωματιδιακής φυσικής, στατιστικής φυσικής και σχετικότητας. Βάσει αυτών και όπως υποδεικνύουν τα φάσματα ενεργών γαλαξιακών πυρήνων, έχουμε μία γενική όψη των συνθηκών και των χωρικών κλιμάκων όπου παράγονται ακτίνες Χ στους \textlatin{AGN}. Έπειτα γίνεται μια σύντομη αναφορά στην δομή και λειτουργία τηλεσκοπίων ακτίνων Χ, ενώ παρουσιάζεται με περισσότερη λεπτομέρεια το τηλεσκόπιο \textlatin{XMM-Newton} και η λήψη δεδομένων από αυτό.

Κεφάλαιο 4: Παρουσιάζονται εν συντομία οι συσχετισμοί μεταβλητότητας σε διαφορετικά μήκη κύματος και οι πιθανές αιτίες της παρατηρούμενης μεταβλητότητας των \textlatin{AGN}. Έπειτα, εξηγείται διεξοδικά πώς προκύπτει και ποσοτικοποιείται η στοχαστική μεταβλητότητα από παρατηρήσεις \textlatin{AGN} στις ακτίνες Χ, ενώ αποσαφηνίζεται η εξαγωγή φασμάτων ενεργών γαλαξιακών πυρήνων από παρατηρησιακά δεδομένα και η σημασία τους για την μέτρηση μεταβλητότητας. Εισάγεται η στατιστική προσέγγιση της μέσης μεταβλητότητας συλλογής, η χρησιμότητά της για έρευνες πληθυσμού και οι προϋποθέσεις για να δώσει ασφαλή αποτελέσματα. Αναφέρονται οι φυσικές παράμετροι των \textlatin{AGN} που σχετίζονται με την παρατηρούμενη μεταβλητότητα. 

Κεφάλαιο 5: Γίνεται επισκόπηση του πεδίου \textlatin{XMM-XXL-N}, περιγράφεται ο πληθυσμός ενεργών γαλαξιακών πυρήνων που πρόκειται να μελετηθεί και οι κατάλογοι που χρησιμοποιήθηκαν. Εξετάζονται τα φωτομετρικά δεδομένα από τους καταλόγους αυτούς και υπολογίζονται οι ποσότητες που αφορούν την έρευνά μας για να συγκρίνουμε το μέγεθος της μεταβλητότητας με την λαμπρότητα των πηγών, περιορίζοντας το δείγμα σύμφωνα με τις προδιαγραφές που θέτουμε ώστε να εξάγουμε χρήσιμα αποτελέσματα. Χαράσονται τα διαγράμματα μεταβλητότητας πηγών και μέσης μεταβλητότητας συλλογής σε σχέση με την λαμπρότητα και ελέγχεται η συσχέτιση.

Κεφάλαιο 6: Παρουσιάζονται οι βασικές αρχές στατιστικής συμπερασματολογίας \textlatin{Bayes} και οι εφαρμογές τους σε δειγματοληπτικούς αλγορίθμους. Για το δείγμα \textlatin{AGN} που περιγράφηκε στο κεφάλαιο 5, προσαρμόζεται αλγόριθμος \textlatin{Bayes} εκτίμησης παραμέτρων ώστε να υπολογιστεί το μέγεθος της μεταβλητότητας σε παραλληλία με τον κλασσικό τρόπο υπολογισμού του από φωτομετρικά μεγέθη παρατηρήσεων που έγινε στο προηγούμενο κεφάλαιο. Συγκρίνονται τα κλασσικά υπολογισμένα μεγέθη με τα αντίστοιχα δειγματοληπτικά και χαράσονται τα διαγράμματα μεταβλητότητας πηγών και μέσης μεταβλητότητας συλλογής όπως προέκυψαν από τον αλγόριθμο \textlatin{Bayes} σε σχέση με την λαμπρότητα και ελέγχεται η συσχέτιση. 

Κεφάλαιο 7: Εφαρμόζεται αλγόριθμος βασισμένος σε παρατηρησιακές σχέσεις που παράγει τεχνητό πληθυσμό \textlatin{AGN} με χαρακτηριστικά προσαρμοσμένα στον πληθυσμό του πεδίου \textlatin{XMM-XXL-N} που μελετάμε. Ο αλγόριθμος αυτός μοντελοποιεί την μεταβλητότητα ενεργών γαλαξιακών πυρήνων σχετίζοντας τα θεμελιώδη μεγέθη του συστήματος προσαύξησης με το αναλυτικό φάσμα με τέσσερεις διαφορετικούς τρόπους, οπότε προκύπτουν τέσσερα διαφορετικά μοντέλα. Χαράσονται τα διαγράμματα μέσης μεταβλητότητας συλλογής με την λαμπρότητα που υπαγορεύουν τα μοντέλα και συγκρίνονται με την σχέση μέσης μεταβλητότητας συλλογής - λαμπρότητας του δείγματός μας όπως υπολογίστηκε με τον κλασσικό τρόπο στο κεφάλαιο 5 και όπως υπολογίστηκε από τον αλγόριθμο \textlatin{Bayes} στο κεφάλαιο 6. Ελέγχεται η προσαρμογή των μοντ'ελων στα δεδομένα μας. 

Κεφάλαιο 8: Σχολιάζονται τα αποτελέσματα της σχέσης μέσης μεταβλητότητας συλλογής με την λαμπρότητα όπως υπολογίστηκαν στο κεφάλαιο 5 και στο κεφάλαιο 6, συγκρίνονται τα συμπεράσματα με μελέτες άλλων πεδίων που έχουν ερευνηθεί, διερευνάται η ασφάλεια των συμπερασμάτων και σχολιάζονται πιθανές αλλαγές και βελτιώσεις της προσέγγισής μας καθώς επίσης και επέκταση για περαιτέρω έρευνα.

Η ανάλυση δεδομένων έγινε με την γλώσσα προγραμματισμού \textlatin{Python 3}.
